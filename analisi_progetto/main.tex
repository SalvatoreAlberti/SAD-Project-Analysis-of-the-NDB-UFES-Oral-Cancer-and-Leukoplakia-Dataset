\documentclass{article}

% Language setting
% Replace `english' with e.g. `spanish' to change the document language
\usepackage[english]{babel}
\usepackage{graphicx}
\usepackage{float}
% Set page size and margins
% Replace `letterpaper' with `a4paper' for UK/EU standard size
\usepackage[letterpaper,top=2cm,bottom=2cm,left=3cm,right=3cm,marginparwidth=1.75cm]{geometry}

% Useful packages
\usepackage{amsmath}
\usepackage{graphicx}
\usepackage[colorlinks=true, allcolors=blue]{hyperref}

\title{Your Paper}
\author{Quaglia Michele, Alberti Salvatore}

\begin{document}
\maketitle

\begin{abstract}

\end{abstract}

\section{EDA}
\subsection{Struttura del dataset}
Il dataset è composto da 237 righe e 17 colonne. Le colonne sono:
\begin{itemize}
    \item public\_id: identificativo per ogni istanza. Intero
    \item lesion\_id: non so a che serve. Intero
    \item patient\_id: identificativo paziente. Intero
    \item path: nome dell’immagine associata all’istanza. Stringa
    \item localization: localizzazione della lesione. Stringa, assume un numero ristretto di valori che sono: {Tongue, Lip, Floor of mouth, Buccal mucosa, Palate, Gingiva}
    \item larger\_size: dimensione della lesione in centimetri. Numerico
    \item tobacco\_use: indica se il paziente fa uso di tabacco. Stringa, può assumere un numero ristretto di valori che sono: {Yes, No, Former, Not informed}, dove former indica un individuo che faceva uso di tabacco in passato mentre not informed che non abbiamo informazioni a riguardo.
    \item alcohol\_consumption: indica se il paziente fa uso di alcol. Stringa, può assumere un numero ristretto di valori che sono: { Yes, No, Former, Not informed}, dove former indica un individuo che faceva uso di alcol in passato mentre not informed che non abbiamo informazioni a riguardo
    \item sun\_exposure: indica se il paziente si espone al sole. Stringa, può assumere un numero ristretto di valori che sono: {Yes, No, Not informed}
    \item gender: indica il genere del paziente. Stringa, può assumere un numero ristretto di valori che sono: {M, F}
    \item skin\_color: indica la carnagione della pelle. Stringa, può assumere un numero ristretto di valori che sono: {Black, White, Not informed}
    \item age\_group; indica la fascia di età del paziente. Intero, può assumere un numero ristretto di valori: {0, 1, 2}, dove 0 è la fascia che va fino ai 40 anni, 1 è la fascia che va dai 41 ai 60, 2 è la fascia che va dai 61 in poi.
    \item diagnosis: è la diagnosi della lesione. Stringa, può assumere un numero ristretto di valori che sono: {OSCC, Leukoplakia with dysplasia, Leukoplakia without dysplasia}
    \item dysplasia\_severity: indica il grado di dysplasia, Stringa, può assumere un numero ristretto di valori che sono: {Mild, Severe, Moderate}
    \item TaskII: etichetta diagnostica binaria definita dagli autori del dataset per il Task II di classificazione. A partire dalla diagnosi istopatologica dettagliata (diagnosis), i casi vengono ricodificati in due sole categorie di Stringa: {OSCC, Leukoplakia}
    \item TaskIII: etichetta diagnostica multiclass definita per il Task III di classificazione.
    A differenza di TaskII, che raggruppa tutte le leucoplachie in un’unica categoria, TaskIII le distingue in base a se è presente o meno dysplasia, in caso di OSCC viene considerata presente. Stringa, può assumere 2 valori: {Presence, Absence}
    \item TaskIV: etichetta diagnostica multiclass che distingue tutte le condizioni di TaskIII ma che può assumere tutti i valori possibili che sono: {OSCC, Leukoplakia with dysplasia, Leukoplakia without dysplasia}
    
    
\end{itemize}
\subsection{Valori nulli o mancanti}
Il dataset non contiene valori nulli, inoltre l’unica colonna che contiene valori mancanti è dysplasia\_severity, con 148 valori mancanti su 237 totali.
\subsection{Analisi univariata}
\paragraph{Variabili quantitative}
L’unica variabile quantitativa è larger\_size.
\subsection{larger\_size}
\begin{table}[H]
    \centering
    \caption{Statistiche descrittive di \texttt{larger\_size}}
    \begin{tabular}{lrrrrrr}
        \hline
                & Min. & 1st Qu. & Median & Mean  & 3rd Qu. & Max. \\
        \hline
                  0.00 & 4.00 & 15.00 & 16.48 & 25.00  & 60.00 \\
        \hline
    \end{tabular}

    \vspace{0.5em}
    Deviazione standard: 14.35228
\end{table}

\begin{figure}[H]
    \centering
    \includegraphics[width=0.7\textwidth]{grafici/istogramma_larger_size.png}
    \caption{Come è possibile notare dall’istogramma, i valori sono maggiormente concentrati tra 0 e 15, cioè a sinistra della mediana, mentre a destra della mediana sono più rari}
    \label{fig:larger_size}
\end{figure}

\begin{figure}[H]
    \centering
    \includegraphics[width=0.7\textwidth]{grafici/densità_larger_size.png}
    \caption{A sostegno di ciò, nel grafico di densità, i valori massimi si raggiungono nell’intervallo che va tra 0 e 15, con una successiva decrescenza.}
    \label{fig:larger_size}
\end{figure}

\begin{figure}[H]
    \centering
    \includegraphics[width=0.7\textwidth]{grafici/boxplot_larger_size.png}
    \caption{Il boxplot evidenzia la presenza di un valore outlier, cioè 60.}
    \label{fig:larger_size}
\end{figure}

\paragraph{Variabili qualitative}
Le variabili qualitative sono: localization, tobacco\_use, alcohol\_consumption, sun\_exposure, gender, skin\_color, age\_group", diagnosis, dysplasia\_severity, TaskII, TaskIII, TaskIV.
\subsection{localization}
\begin{table}[H]
    \centering
    \caption{Frequenze assolute e relative per localization}
    \begin{tabular}{lrr}
        \hline
           &            Frequenza assoluta & Frequenza relativa (\%) \\
        \hline
        Buccal mucosa    & 35 & 14.77 \\
        Floor of mouth   & 32 & 13.50 \\
        Gingiva          & 59 & 24.89 \\
        Lip              & 20 & 8.44  \\
        Palate           & 12 & 5.06  \\
        Tongue           & 79 & 33.33 \\
        \hline
        Totale           & 237 & 100.00 \\
        \hline
    \end{tabular}
\end{table}

\begin{figure}[H]
    \centering
    \includegraphics[width=0.7\textwidth]{grafici/barplot_localization.png}
    \caption{barplot per localization}
    \label{fig:larger_size}
\end{figure}

Le analisi effettuate mostrano che la localizzazione più frequente è la lingua, che corrisponde ad un terzo  dei dati, segue subito gengiva che corrisponde a un quarto. Le altre localizzazioni sono meno frequenti.
\subsection{tobacco\_use}
\begin{table}[H]
    \centering
    \caption{Frequenze assolute e relative per tobacco\_use}
    \begin{tabular}{lrr}
        \hline
           &             Frequenza assoluta & Frequenza relativa (\%) \\
        \hline
        Former    & 14 & 5.907173  \\
        No         & 43 & 18.143460 \\
        Not informed         & 112 & 47.257384  \\
        Yes              & 68 & 28.691983  \\
        \hline
        Totale           & 237 & 100.00 \\
        \hline
    \end{tabular}
\end{table}

\begin{figure}[H]
    \centering
    \includegraphics[width=0.7\textwidth]{grafici/barplot_tobacco_use.png}
    \caption{barplot per tobacco\_use}
    \label{fig:larger_size}
\end{figure}
Per il 47\% dei pazienti non abbiamo informazioni  riguardo dell’uso di tabacco, proviamo ad effettuare un'analisi escludendo questi individui:
\begin{table}[H]
    \centering
    \caption{Frequenze assolute e relative per tobacco\_use senza considerare "not informed"}
    \begin{tabular}{lrr}
        \hline
           &             Frequenza assoluta & Frequenza relativa (\%) \\
        \hline
        Former    & 14 & 11.2     \\
        No         & 43 & 34.4       \\
        Yes              & 68 & 54.4    \\
        \hline
        Totale           & 125 & 100.00 \\
        \hline
    \end{tabular}
\end{table}

\begin{figure}[H]
    \centering
    \includegraphics[width=0.7\textwidth]{grafici/barplot_tobacco_use_senza_notinformed.png}
    \caption{barplot per tobacco\_use senza considerare "not informed"}
    \label{fig:larger_size}
\end{figure}
Tra i pazienti per i quali disponiamo di informazioni riguardo all’uso di tabacco, poco più della metà ne fa uso (54.4\%) mentre l'11.2\% ne faceva uso, quindi il 65.6\% ne ha fatto uso almeno una volta. Il restante 34.4\% non ha mai fatto uso di tabacco.

\subsection{alcohol\_consumption}

\begin{table}[H]
    \centering
    \caption{Frequenze assolute e relative per alcohol\_consumption}
    \begin{tabular}{lrr}
        \hline
           &             Frequenza assoluta & Frequenza relativa (\%) \\
        \hline
        Former    & 28 & 11.81435  \\
        No         & 63 & 26.58228 \\
        Not informed         & 112 & 47.25738  \\
        Yes              & 34 & 14.34599   \\
        \hline
        Totale           & 237 & 100.00 \\
        \hline
    \end{tabular}
\end{table}

\begin{figure}[H]
    \centering
    \includegraphics[width=0.7\textwidth]{grafici/barplot_alcohol_consumption.png}
    \caption{barplot per alcohol\_consumption}
    \label{fig:larger_size}
\end{figure}

Come per tobacco\_use, anche in questo caso per il 47\% dei pazienti non abbiamo informazioni a riguardo dell’uso di alcol, quindi proviamo ad effettuare un'analisi escludendo questi individui

\begin{table}[H]
    \centering
    \caption{Frequenze assolute e relative per alcohol\_consumption senza considerare "not informed"}
    \begin{tabular}{lrr}
        \hline
           &             Frequenza assoluta & Frequenza relativa (\%) \\
        \hline
        Former    & 28 & 22.4        \\
        No         & 63 & 50.4          \\
        Yes              & 34 & 27.2     \\
        \hline
        Totale           & 125 & 100.00 \\
        \hline
    \end{tabular}
\end{table}

\begin{figure}[H]
    \centering
    \includegraphics[width=0.7\textwidth]{grafici/barplot_alcohol_consumption_senza_notinformed.png}
    \caption{barplot per alcohol\_consumption senza considerare "not informed"}
    \label{fig:larger_size}
\end{figure}
Tra i pazienti per i quali disponiamo di informazioni riguardo all’uso di alcol, la metà non ne ha mai fatto uso (50,4\%), mentre il 27,2\% ne fa un uso abituale e il 22,4\% ha cessato il consumo, pertanto il 49.6\% ne ha fatto uso almeno una volta.

\subsection{sun\_exposure}

\begin{table}[H]
    \centering
    \caption{Frequenze assolute e relative per sun\_exposure}
    \begin{tabular}{lrr}
        \hline
           &             Frequenza assoluta & Frequenza relativa (\%) \\
        \hline
        No         & 80 &     33.75527      \\
        Not informed         & 122 & 51.47679       \\
        Yes              & 35 & 14.76793    \\
        \hline
        Totale           & 237 & 100.00 \\
        \hline
    \end{tabular}
\end{table}

\begin{figure}[H]
    \centering
    \includegraphics[width=0.7\textwidth]{grafici/barplot_sun_exposure.png}
    \caption{barplot per sun\_exposure}
    \label{fig:larger_size}
\end{figure}
Anche in questo caso, non disponiamo informazioni sul 51.47\% degli individui, effettuiamo un'analisi escludendoli.

\begin{table}[H]
    \centering
    \caption{Frequenze assolute e relative per sun\_exposure senza considerare "not informed"}
    \begin{tabular}{lrr}
        \hline
           &             Frequenza assoluta & Frequenza relativa (\%) \\
        \hline
        No         & 80 & 69.56522           \\
        Yes              & 35 & 30.43478      \\
        \hline
        Totale           & 115 & 100.00 \\
        \hline
    \end{tabular}
\end{table}

\begin{figure}[H]
    \centering
    \includegraphics[width=0.7\textwidth]{grafici/barplot_sun_exposure_senza_notinformed.png}
    \caption{barplot per sun\_exposure senza considerare "not informed"}
    \label{fig:larger_size}
\end{figure}
Tra i pazienti per i quali disponiamo di informazioni riguardo all’esposizione al sole, circa il 70\% non si espone mentre il 30\% si.


\subsection{gender}

\begin{table}[H]
    \centering
    \caption{Frequenze assolute e relative per gender}
    \begin{tabular}{lrr}
        \hline
           &             Frequenza assoluta & Frequenza relativa (\%) \\
        \hline
        M         & 105 &     55.6962       \\
        F              & 132 & 44.3038     \\
        \hline
        Totale           & 237 & 100.00 \\
        \hline
    \end{tabular}
\end{table}

\begin{figure}[H]
    \centering
    \includegraphics[width=0.7\textwidth]{grafici/barplot_gender.png}
    \caption{barplot per gender}
    \label{fig:larger_size}
\end{figure}
I pazienti di genere maschile sono un pò più numerosi rispetto ai pazienti di genere femminile

\subsection{skin\_color}

\begin{table}[H]
    \centering
    \caption{Frequenze assolute e relative per skin\_color}
    \begin{tabular}{lrr}
        \hline
           &             Frequenza assoluta & Frequenza relativa (\%) \\
        \hline
        Black         & 31 &     13.080169           \\
        Brown        & 17 &     7.172996          \\
        Not informed         & 100 & 42.194093           \\
        White              & 89 & 37.552743     \\
        \hline
        Totale           & 237 & 100.00 \\
        \hline
    \end{tabular}
\end{table}

\begin{figure}[H]
    \centering
    \includegraphics[width=0.7\textwidth]{grafici/barplot_skin_color.png}
    \caption{barplot per skin\_color}
    \label{fig:larger_size}
\end{figure}
Non abbiamo informazioni per il 42\% dei pazienti, pertanto conduciamo un'analisi senza considerare questi

\begin{table}[H]
    \centering
    \caption{Frequenze assolute e relative per skin\_color senza considerare "not informed"}
    \begin{tabular}{lrr}
        \hline
           &             Frequenza assoluta & Frequenza relativa (\%) \\
        \hline
        Black         & 31 &     22.62774          \\
        Brown        & 17 &     12.40876           \\
        White              & 89 & 64.96350     \\
        \hline
        Totale           & 137 & 100.00 \\
        \hline
    \end{tabular}
\end{table}

\begin{figure}[H]
    \centering
    \includegraphics[width=0.7\textwidth]{grafici/barplot_skin_color_senza_notinformed.png}
    \caption{barplot per skin\_color senza considerare "not informed"}
    \label{fig:larger_size}
\end{figure}
La maggior parte dei pazienti ha carnagione chiara (65\%), seguono i pazienti con "carnagione nera" (22.62\%) e per ultimi i pazienti con "carnagione marrone" (12.4\%).

\subsection{age\_group}

\begin{table}[H]
    \centering
    \caption{Frequenze assolute e relative per age\_group}
    \begin{tabular}{lrr}
        \hline
           &             Frequenza assoluta & Frequenza relativa (\%) \\
        \hline
        Gruppo 0         & 15 &     6.329114        \\
        Gruppo 1              & 111 & 46.835443      \\
        Gruppo 2              & 111 & 46.835443      \\
        \hline
        Totale           & 237 & 100.00 \\
        \hline
    \end{tabular}
\end{table}

\begin{figure}[H]
    \centering
    \includegraphics[width=0.7\textwidth]{grafici/barplot_age_group.png}
    \caption{barplot per age\_group}
    \label{fig:larger_size}
\end{figure}
I pazienti di età inferiore ai 40 anni sono in netta minoranza (6.32\%), mentre i pazienti tra i 41 e 60 ed i pazienti dai 61 anni in su sono in parità (46.83\%)

\subsection{diagnosis}

\begin{table}[H]
    \centering
    \caption{Frequenze assolute e relative per diagnosis}
    \begin{tabular}{lrr}
        \hline
           &             Frequenza assoluta & Frequenza relativa (\%) \\
        \hline
           Leukoplakia with dysplasia          & 89 &     37.55274         \\
        Leukoplakia without dysplasia              & 57 & 24.05063      \\
        OSCC              & 91 & 38.39662      \\
        \hline
        Totale           & 237 & 100.00 \\
        \hline
    \end{tabular}
\end{table}
\begin{figure}[H]
    \centering
    \includegraphics[width=0.7\textwidth]{grafici/barplot_diagnosis.png}
    \caption{barplot per diagnosis}
    \label{fig:larger_size}
\end{figure}
Dalle analisi emerge che i pazienti affetti da OSCC sono in leggera maggioranza (38.39\%) rispetto ai pazienti affetti da Leukoplakia con dysplasia (37.55\%). I pazienti affetti da Leukoplakia che non presentano dysplasia sono in numero inferiore (24\%)

\subsection{dysplasia_severity}
\begin{table}[H]
    \centering
    \caption{Frequenze assolute e relative per dysplasia\_severity}
    \begin{tabular}{lrr}
        \hline
           &             Frequenza assoluta & Frequenza relativa (\%) \\
        \hline
        NA        & 148 &     62.447257  
         Mild         & 58 &     24.472574           \\
        Moderate             & 15 & 6.329114        \\
        Severe              & 16 & 6.751055       \\
        \hline
        Totale           & 237 & 100.00 \\
        \hline
    \end{tabular}
\end{table}

\begin{figure}[H]
    \centering
    \includegraphics[width=0.7\textwidth]{grafici/barplot_dysplasia_severity.png}
    \caption{barplot per dysplasia\_severity}
    \label{fig:larger_size}
\end{figure}
Il 62\% dei valori è vuoto, pertanto li poniamo NA e analizziamo i soli valori non NA:

\begin{table}[H]
    \centering
    \caption{Frequenze assolute e relative per dysplasia\_severity senza considerare i valori NA}
    \begin{tabular}{lrr}
        \hline
           &             Frequenza assoluta & Frequenza relativa (\%) \\
        \hline
         Mild         & 58 &     65.16854            \\
        Moderate             & 15 & 16.85393         \\
        Severe              & 16 & 17.97753        \\
        \hline
        Totale           & 89 & 100.00 \\
        \hline
    \end{tabular}
\end{table}

\begin{figure}[H]
    \centering
    \includegraphics[width=0.7\textwidth]{grafici/barplot_dysplasia_severity_senzaNA.png}
    \caption{barplot per dysplasia\_severity senza considerare i valori NA}
    \label{fig:larger_size}
\end{figure}
Tra i pazienti di cui abbiamo informazioni a riguardo, il 65\% ha una dysplasia lieve, il 16.85\% moderata e circa il 18\% severa.

\subsection{TaskII}

\begin{table}[H]
    \centering
    \caption{Frequenze assolute e relative per TaskII}
    \begin{tabular}{lrr}
        \hline
           &             Frequenza assoluta & Frequenza relativa (\%) \\
        \hline
        Leukoplakia         & 146 &  61.60338            \\
        OSCC              & 91 & 38.39662      \\
        \hline
        Totale           & 237 & 100.00 \\
        \hline
    \end{tabular}
\end{table}

\begin{figure}[H]
    \centering
    \includegraphics[width=0.7\textwidth]{grafici/barplot_taskII.png}
    \caption{barplot per TaskII}
    \label{fig:larger_size}
\end{figure}
La maggior parte dei pazienti presenta una Leukoplakia (61.6\%), mentre il 38.39\% è affetto da OSCC

\subsection{TaskIII}
\begin{table}[H]
    \centering
    \caption{Frequenze assolute e relative per TaskIII}
    \begin{tabular}{lrr}
        \hline
           &             Frequenza assoluta & Frequenza relativa (\%) \\
        \hline
        Absence         & 57 &  24.05063             \\
        Presence             & 180 & 75.94937      \\
        \hline
        Totale           & 237 & 100.00 \\
        \hline
    \end{tabular}
\end{table}

\begin{figure}[H]
    \centering
    \includegraphics[width=0.7\textwidth]{grafici/barplot_taskIII.png}
    \caption{barplot per TaskIII}
    \label{fig:larger_size}
\end{figure}
Circa il 76\% dei pazienti presenta una dysplasia, quindi è affetto da OSCC oppure da una Leukoplakia con dysplasia.

\subsection{TaskIV}

\begin{table}[H]
    \centering
    \caption{Frequenze assolute e relative per TaskIV}
    \begin{tabular}{lrr}
        \hline
           &             Frequenza assoluta & Frequenza relativa (\%) \\
        \hline
        Leukoplakia with dysplasia         & 89 &  37.55274             \\
        Leukoplakia without dysplasia             & 57 & 24.05063      \\
        OSCC             & 91 & 38.39662       \\
        \hline
        Totale           & 237 & 100.00 \\
        \hline
    \end{tabular}
\end{table}

\begin{figure}[H]
    \centering
    \includegraphics[width=0.7\textwidth]{grafici/barplot_taskIV.png}
    \caption{barplot per TaskIV}
    \label{fig:larger_size}
\end{figure}
I pazienti affetti da OSCC sono leggermente in maggioranza (38.39\%) sui pazienti affetti da una Leukoplakia con dysplasia (37.55\%). I pazienti che presentano una leukoplakia senza dysplasia rappresentano la minoranza (24\%)
\end{document}
